%% LyX 1.4.3 created this file.  For more info, see http://www.lyx.org/.
%% Do not edit unless you really know what you are doing.
\documentclass[a4paper, english]{article}
 \usepackage{graphicx}
 \usepackage{amssymb}
 \usepackage{epstopdf}
 \usepackage[usenames]{color}
 
\IfFileExists{url.sty}{\usepackage{url}}
                      {\newcommand{\url}{\texttt}}
                      
                      

%%%%%%%%%%%%%%%%%%%%%%%%%%%%%% LyX specific LaTeX commands.
\newcommand{\noun}[1]{\textsc{#1}}
%% Bold symbol macro for standard LaTeX users
\providecommand{\boldsymbol}[1]{\mbox{\boldmath $#1$}}

%%%%%%%%%%%%%%%%%%%%%%%%%%%%%% Textclass specific LaTeX commands.
\newenvironment{lyxlist}[1]
{\begin{list}{}
{\settowidth{\labelwidth}{#1}
 \setlength{\leftmargin}{\labelwidth}
 \addtolength{\leftmargin}{\labelsep}
 \renewcommand{\makelabel}[1]{##1\hfil}}}
{\end{list}}

%%%%%%%%%%%%%%%%%%%%%%%%%%%%%% User specified LaTeX commands.

\definecolor{XMLColor}{rgb}{0.0,0.0,0.0} 
\newcommand{\xml}[1]{\textcolor{XMLColor}{\texttt{#1}}}

\usepackage{babel}
\makeatother
\begin{document}

\title{\textbf{The Yule model}}

\author{\textsc{Alexei J. Drummond}}

\maketitle

\section{Probability density for Yule model}

The Yule model of branching is a pure birth process in which each branch has associated with it a birth rate ($\lambda$) determining the instantaneous rate at which the branch gives birth to a new branch (bifurcates into two branches). Starting at the root of a tree (the first bifurcation), there are two descendant branches, each with a birth rate of $\lambda$. This gives rise to the following probability density function for the time to the second bifurcation ($t_2$):

\begin{equation}
p(t_2|\lambda) = 2\lambda e^{-2\lambda t_2}
\end{equation}

This is simply an exponential distribution with a mean of $\frac{1}{2\lambda}$. Likewise the probability density for the time from the $(i-1)$th to the $i$th bifurcation ($t_i$) is:

\begin{equation}
p(t_i|\lambda) = i\lambda e^{-i\lambda t_i}
\end{equation}

So the probability density for a tree that has {\it just reached} the $n$th bifurcation is:

\begin{equation}
q(\mathbf{t}|\lambda,n) = \prod_{i=2}^{n-1} i\lambda e^{-i\lambda t_i}
\end{equation}

However most trees are not sampled exactly at the moment of the final bifurcation event. This means there is a final time $t_n$ over which {\it none} of the $n$ branches bifurcated. The probability of waiting $t_n$ time without seeing any of the $n$ branches bifurcate is $e^{-n\lambda t_n}$ giving rise to a total probability of a tree of $n$ tips:

\begin{equation}
p(\mathbf{t}|\lambda,n) = (n-1)!\lambda^{n-2}\prod_{i=2}^{n} e^{-i\lambda t_i}
\end{equation}

Remembering that $e^ae^b = e^{a+b}$ and defining the total tree length $s = \sum_{i=2}^{n} it_i$ we have:

\begin{equation}
p(\mathbf{t}|\lambda,n) = (n-1)!\lambda^{n-2}e^{-\lambda s}
\end{equation}

This probability density is the same as that of equation (3) in Nee {\it et al} (2001), despite Nee making the confusing assertion that his equation (3) is conditional on the root height. From the construction above, it is clear that no such conditioning exists.

\section{Expectation of the tree height and total tree length under the Yule model}

Defining the total tree height, $t_{MRCA} = \sum_{i=2}^{n} t_i$, it is easy to show that the expected tree height is:

 \begin{equation}
E(t_{MRCA}) = \sum_{i=2}^{n} \frac{1}{i\lambda}
\end{equation}

So for a 4 taxon tree and $\lambda=2$ the expected height is:

 \begin{equation}
 E(t_{MRCA}) = \frac{1}{4} + \frac{1}{6} + \frac{1}{8} =  0.5416666
\end{equation}

The total expected tree length ($s$) of a Yule branching process is:

 \begin{equation}
E(s) = \sum_{i=2}^{n} \frac{1}{\lambda} = \frac{n-1}{\lambda}
\end{equation}

So for a 4 taxon tree and $\lambda=2$ the expected height is:

 \begin{equation}
 E(s) = \frac{3}{2.0} =  1.5
\end{equation}
 
\end{document}

